% Todos:
% features.txt in the root directory

\usepackage[english]{babel}
\usepackage[utf8]{inputenc}
\usepackage[T1]{fontenc}

\usepackage{graphicx}
\usepackage{hyperref}
\usepackage{wallpaper}
\usepackage{xcolor}
\usepackage{geometry}
\usepackage{pdflscape}
\usepackage{gensymb}

\pagestyle{headings}

\begin{document}
\newgeometry{bottom=2cm}
\ThisCenterWallPaper{1.0}{images/title}
\begin{titlepage}
  \centering
  \sf
  {\Huge Saab \ifthenelse{\boolean{AJS}}{AJS 37}{JA 37Di} Viggen}
  \\[1cm]
  {\Huge Flightgear Flight Manual}
  \\[16cm]
  \color{white}
  \emph{model by}\\
  Anders Lejczak, Justin Nicholson, Enrico Castaldi,\\
  Joshua Davidson, Nicola B.\ Bernardelli, Isaac Protiva,\\
  Nikolai V.\ Chr.
\end{titlepage}
\restoregeometry

\currentpdfbookmark{contents}{Contents}
\tableofcontents

\chapter{Introduction}
\currentpdfbookmark{introduction}{Introduction}
\addcontentsline{toc}{chapter}{Introduction}

Welcome! You are reading the \ifthenelse{\boolean{AJS}}{AJS 37}{JA 37Di} version of this manual. There is a similar manual for the \ifthenelse{\boolean{AJS}}{JA 37Di}{AJS 37} variant of the Viggen.

This guide intends to be a pilot's handbook for the Saab 37 Viggen aircraft modelled for the \href{http://www.flightgear.org}{FlightGear (FG)} flight simulator. The manual assumes acquaintance with FlightGear and does in general not provide any generic help to use the simulator.

The original aircraft's objective was military combat --- and that is also an explicit part of this manual.

The Saab Viggen was originally intended as a multi-role combat aeroplane. However, while largely a sngle type of aircraft/airframe was able to fulfill multiple roles, it was only the Viggen's Swedish successor Saab JAS 39 Gripen that was able to fulfill the requirements for a multi-purpose combat airplane. Consequently, the Viggen type designations reflect their operational roles:
\begin{description}
\item[AJ] Attack, Jakt (surface attack with secondary air defence role)
\item[SK] Skol (conversion training)
\item[SH] Spaning, Havövervakning (reconnaissance, sea/naval surveillance)
\item[SF] Spaning, Foto (reconnaissance, photography)
\item[JA] Jakt, Attack (air defence with secondary attack role)
\end{description}

The Viggen variants explicitly modelled in FlighGear are the most modern variant for the fighter respectively the attack variant:
\begin{itemize}
 \item the \emph{JA 37Di} featuring some of the glass instrument panels used in the JAS 39 Gripen --- \texttt{Di} for digital. 
 \item the \emph{AJS 37} which resulted out of a modification programme providing existing AJ 37, SF 37 and SH 37 variants with a limited multi-role (attack, fighter and reconnaissance) capability.
 \end{itemize}
 
 Note that there also is a AJ 37 variant modelled, however it is not keps ut-to-date and does not have the developers' focus.
 
 In the root directory of the Viggen aircraft in Flightgear (e.g. in something like \emph{flightgear-saab-ja-37-viggen/Aircraft/JA37}) there is a file called \emph{features.txt}. Over time all these features will be described in this manual --- until then please have a look at that text file to get a feeling for the impressive amount of features implemented (especially in the JADi variant).

\section{References}

\subsection{Original Manuals}
A significant part of the original manuals are available on the internet, and in recent years also many of the formerly classified chapters (as far as the information is not still valid due to (re-)use in the Gripen) have been become available. As the Viggen could not be sold to other countries\footnote{Austrian pilots went through a training programme on the Viggen to familiarise with modern combat airplane, but never bought it}, a significant part of the text and illustrations are in Swedish.

The following manuals are core to the understanding of flying a Viggen in military operations. Where possible, references are made in this manual to text elements in the original manuals(e.g. [AJS\_Del1, sec. 20, ch. 3.4.2, p. 312]). Also, this manual does typically not repeat illustrations from the original handbooks, but shows a screenshot from the simulation instead, so you can compare.

\begin{table}[!th]
\begin{tabular}{|l|l|c|r|}
\hline
Ref & Title & Date & Pages \\
\hline
AJS\_Del1 & FPL AJS37 Speciell Förarinstruktion Del 1 (M5800--370011) & 1994--11--15 & 517 \\
AJS\_Del2 & FPL AJS37 Speciell Förarinstruktion Del 2 (M5800--370011) & 1994-11-01 & 222 \\
AJS\_Del3 & FPL AJS37 Speciell Förarinstruktion Del 3 (M5800--370011) & 1994-11-15 & 295 \\
JA\_Vol1 & Flight Manual A/C JA37 Volume 1 (M5800-370051) & 1999--01--20 & 497 \\
\hline
\end{tabular}
\caption{Overview of flight manuals from the original Viggen}
\end{table}

\subsection{FlightGear Related}
\begin{itemize}
\item The \href{http://wiki.flightgear.org/Saab_37_Viggen}{FG wiki article} contains a compreensive feature overview, links to forum articles etc. However, the wiki is not always up to date.
\item \href{http://opredflag.com/}{Operation Red Flag (OPRF)} is a FlightGear military simulation community discussing, developing and using many of the combat features in the Viggen.
\item On Discord there is a dedicated \href{https://discord.gg/jc5pSM5}{Viggen server} and a \href{https://discord.gg/SmGFnJN}{OPRF server}. Say hello and discuss the Viggen's features, development and your flight experience. Preliminary help with issues can be gotten as well, confirmed issues should be reported in the Viggen's \href{https://github.com/NikolaiVChr/flightgear-saab-ja-37-viggen/issues}{issue tracker} for resolution.
\item The \href{https://www.youtube.com/playlist?list=PLogi97V-ki0GfCLqimTtIq9RIVcm-GRFE}{Flightgear Saab 37 Viggen YouTube Playlist} includes amongst others a set of tutorial videos featuring the JA and the AJS variant.
\end{itemize}

\subsection{Other Combat Flight Simulators}
Both \href{https://www.digitalcombatsimulator.com/en/index.php}{DCS} and \href{https://www.benchmarksims.org/}{BMS Falcon} have flyable Viggens. Especially for DCS there is an abundance of resources available on the internet (e.g. YouTube). Please note that while there can be valuable additional information and context provided in content from other simulators, the capabilities modelled will differ.

\subsection{Swedish Documentation Sites}
\begin{itemize}
\item Arboga Elektronikhistoriska Förening: \url{https://www.aef.se}, e.g. 
\begin{itemize}
\item \href{https://www.aef.se/Avionik/Notiser/PS-37/PS-37A.htm}{Spanings- och siktesradar PS-37/A (for the AJS)}
\item \href{https://www.aef.se/Avionik/Notiser/Siktesradar_PS-46_1.htm}{Flygradarinstallation JA37 ---Siktesradar PS46} (includes links to 2 articles from Ericsson)
\item \href{https://www.aef.se/Flygvapnet/Tidskrifter/FV_Nytt/FVN_oversikt.htm}{FlygvapenNytt 1960 – 2003}: \url{https://www.aef.se/Flygvapnet/Tidskrifter/FV_Nytt/Flygvapennytt_1991-2.pdf} announced the AJS Viggen and \url{https://www.aef.se/Flygvapnet/Tidskrifter/FV_Nytt/Flygvapennytt_1994-4.pdf} had follow-up articles.
\end{itemize}
\item Digitalt Museum: \url{https://digitaltmuseum.se/}:
\begin{itemize}
\item Aeroseum Göteborg: \url{https://digitaltmuseum.se/owners/S-AER}
\item Flygvapenmuseum: \url{https://digitaltmuseum.se/owners/S-FV}
\item Teknikland: \url{https://digitaltmuseum.se/owners/S-TL}
\item Sveriges militärhistoriska arv: \url{https://digitaltmuseum.se/owners/S-MHA}
\end{itemize}
\end{itemize} 


\chapter{Generic FlightGear Operations}
\section{Download and Installation}
The JA-37Di and the AJS-37 variants of the Viggen are downloaded and installed as one set. You can choose between two options to download and install:
\begin{itemize}
\item The \href{https://github.com/NikolaiVChr/flightgear-saab-ja-37-viggen}{Git repository} has the latest version of the Viggen and is the preferred way to keep up with the constant evolution of the FlightGear model. If you are using the Git repository, make sure to add the correct directory path\footnote{E.g. /home/pingu/fg\_aircraft/flightgear-saab-ja-37-viggen/Aircraft} to the \emph{Additional aircraft folders} section in tab \emph{Add-ons} in the FlightGear launcher.
\item FlightGear's official hangar \href{http://wiki.flightgear.org/FGAddon}{FGAddon} is peridically updated with stable versions from the Git version. 
\end{itemize}

The guide you are reading is keeping up with the latest Git version and therefore the features available in the FGAddon version might be different. Unless you have a good reason not to do so, please consider using the Git version and keep it updated.

In the \emph{Aircraft} tab in the Launcher make sure to choose the correct variant for your next flight.

\section{Compatibility}
The Viggen is tested against FlightGear version 2018.3.x is tested. Earlier version will probably not work, later might.

Rendering framework in FlightGear: The aircraft is sort of \href{http://wiki.flightgear.org/Project_Rembrandt}{Rembrandt} ready without glaring issues. \href{http://wiki.flightgear.org/Atmospheric_light_scattering}{ALS} is actively maintained and extended in the Viggen and therefore recommended.

\section{Menu Entries and In-Flight Configuration}
\subsection{Menu Help}
The Help menu contains 2 menu items, which are Viggen specific:
\begin{itemize}
\item Aircraft Help: the top section contains the most important key bindings for the Viggen\footnote{The displayed list is only a selection. For the total list you need to delve into the installation directory and look into flightgear-saab-ja-37-viggen/Aircraft/JA37/Systems/ja37-input.xml.}. The lower section contains a set of tipps to get your started with flying.
\item Aircraft Checklists: A set of interactive checklists especially for start-up, taxi, take-off and landing.
\end{itemize}

\subsection{Menu AJS-37}
This menu contains a set of menu items, which ease some configurations, which are not available otherwise or would be tedious to do.
\begin{itemize}
\item Select Livery: there is a vairety of liveries available, some of which are fictional. Note that the dialog does not take care of whether a livery belongs to the AJS or JADi version: if you want to fly authentic, you need to know yourself.
\item Auto start/stop: Lets you start and stop the plane. The progress is shown in the top centre of the screen in blue text. \glqq Engine ready\grqq is the final notification of the start-up sequence. The shut-down sequence is done, when the aircraft is dark.
\item Repair
\item Loadout: Opens a custom dialog to allow fast selection of a suitable loadout. This is an alternative to menu item Fuel and Payload in menu Equipment. It will make sure that you only select weapon combinations, which were valid in the real Viggen.
\item Refule: Standard: all tanks full, but no external centre tank
\item Refule: Long distance: adds fuel to the drop tank on the centre fuselage pylon.
\item Refule: Test flight: reduced fuel amount to ca. 80\% of the standard
\item Performance Monitor
\item Systems Monitor
\item Toggle external power: note that this is one of the things added and removed automatically when using the auto start/stop function.
\item Options: various configuration options
\item MP Options: use this dialog to enable/disable weapons messages over MP.
\end{itemize}

\subsection{Menu View}
In menu item \emph{Rendering Options}:
\begin{itemize}
\item Enable \emph{Atomospheric Light Scattering (ALS)} on the right side in \emph{Shader Effects}. Select \emph{Custom settings (fine tuning)} and in \emph{Shader Options} set the slider related to \emph{Model} to the right side (highest setting). This is amonst otheres needed to get nice flood lights in the cockpit at night.
\item Enable Particles
\item Enable Precipitation (you get nice rain drops on the canopy when it rains)
\item 3D clouds: on
\end{itemize}

The \emph{View options} menu item allows to show tooltips on mouse over / mouse press for gauges and knobs.

In menu item \emph{Cockpit View Options}
\begin{itemize}
\item Disable \glqq Dynamic cockpit view \grqq
\item Disable \glqq View movement due to G-Force\grqq
\item Enable \glqq Blackout and Redout due to G-force\grqq. It will be enabled automatically if you enable discharge messages.
\end{itemize}

In menu item \emph{Adjust LOD Ranges}, section \emph{AI/MP aircraft}: select \glqq High Detail only\grqq if you use FG version 2019.1 or higher, else set \glqq Use detailed MP models\grqq.

\subsection{Weather in Menu Environment}
Enable \emph{Detailed Weather} and press \emph{Advanced Settings ...} to set the following options (after all: the Viggen is an all-weather fighter and can handle it --- if your computer hardware can handle it):
\begin{itemize}
\item Wind-model: aloft-waypoints
\item Terrain presampling
\item Realistic visibility
\end{itemize}

\part{Aircraft Presentation}
This manual does not contain a generic presentation of the real Viggen. There are many books and internet resources, which do a good job for this. A comprehensive book in English covering all variants is \glqq Saab 37 Viggen --- The ultimate portfolio\grqq by Jan Jørgensen (published in 2014 by \href{http://www.nordicairpower.com/}{Nordic Airpower}).

What is presented her is how the aircraft is modelled in FlightGear.

\chapter{Aircraft Overview}
\chapter{Controls and Instrumentation}
\chapter{Systems}
% TODO: https://github.com/NikolaiVChr/flightgear-saab-ja-37-viggen/issues/120
%\section{Course Correction for Magnetic Deviation}
%Like in the real Viggen there are two ways to set the course correction, which makes sure that the instruments show true North instead of magnetic North (e.g. in north Norway the deviation is larger than 12\degree):
%\begin{itemize}
% \item Automatic: Simplified the computer averages the deviation between measured between 110 and 200 km/h during take-off.
% \item Manual: Make sure your airplane is properly aligned with the runway. Press FIXME. The reason for doing it manually is if the runway is slippery or there are strong crosswinds, which might change the way the nose points and therefore give a wrong deviation measurement. 
%\end{itemize}

%Please note that the computer in the real Viggen also makes some comparisons with the stored runway heading(s) to decrease the risk of failures. Warning light \emph{NAV SYST} would lit up. However, this is not modelled in FlightGear. 

%\ifthenelse{\boolean{AJS}}{Reference: [AJS\_1, sec. 20, ch. 3.4, p. 312]}{}

\part{Flight Operation}
\chapter{Standard Procedures}
\chapter{Emergency Procedures}
\chapter{Miscallaneous Notes}
\section{JA37 DI Related}
Be mindful of failure messages in the TI display FAIL menu, if a gear locking mechanism fails due to being deployed at too high speed, that gear will not be able to support the weight of the aircraft till you repair it from the menu.

\part{Weapons Operation}

\chapter{Ground and Naval Attack}
\ifthenelse{\boolean{AJS}}{
\section{Rb 05A missile}
\label{Rb05A}
The Rb 05A was a jam-proof, radio guided super-sonic missile developed in Sweden with similar characteristis as the AGM-65 Maverick. While an optical/IR guidance system was planned, it never got realised because the Maverick was adapted as the Rb 75 instead. The missile was constructed for use against grond and naval targets, but later it was discovered that it could also be used against slow-manoeuvring air targets. 

As the missile was guided visually by the pilot, a smoke free propellant was used and there was a flare in the back. During the first 1.7 seconds of the missile's flight, it was autoguided to fly into the pilot's visual field in fron of the aircraft's nose.

To manually guide the missile in the real Viggen, a mini-joystick on the right console was used with the right hand. Therefore, one of the following was used to free the right hand from the stick:
\begin{itemize}
 \item Take the stick with the left hand and fly the airplane. Reference: [AJS\_2, ch. II, p. 86]
 \item If missile guidance is done during the aircraft flying horisontal, then use autopilot for altitude. If it is done during a light dive, use autopilot for attitude. Reference: [AJS\_3, sec. 7, ch. 5, p. 210]
\end{itemize}

On page 25 in issue 4/1972 of the Swedish magazine \url{https://www.aef.se/Flygvapnet/Tidskrifter/FV_Nytt/Flygvapennytt_1972-4.pdf}{FlygvapenNytt} there is a good description of the capabilities and evolution of the weapon.

Operations (reference: [AJS\_3, sec. 7, ch. 5, p. 210] for decription, [AJS\_2, ch. II, p. 86] for procedure):
\begin{enumerate}
 \item Fly fast and low, navigate to the preprogrammed pop-up point in \emph{NAV} mode.
 \item Before reaching the pop-up point:
 \begin{itemize}
  \item Correct the atmospheric pressure.
  \item Choose mode \emph{ANF} (not yet modelled correctly in FlightGear).
  \item Set the \emph{J/A} weapons knob to Rb 05A with the correct fuse (MARK for ground, SJÖ for naval). Not yet modelled correctly in FlightGear: instead choose the Rb 05A by pressing \textbf{\texttt{w}} until displayed on the HUD.
  \item Check that the autopilot is on \emph{SPAK} only.
 \end{itemize}
 \item A fast pop-up is done to 300-400 metres above ground at the pre-programmed pop-up point --- either with pitch only or in a half roll.
 \item Identify the target and fly level or slightly downwards. The nose does not have to point to the target. 
 \item Consider engaging the \emph{ATT} or \emph{ALT} autopilot mode.
 \item Fire within 9 km of the target at a speed between 700 km/h and 1150 km/h.
 \item Guide the missile visually using the collimation principle --- to begin with roughly and then more and more accurate until the missile' flare covers the target.
 \item When the missile has hit the target, evade as quickly as possible and choose \emph{NAV} mode.
\end{enumerate}

Operations in FlightGear:
\begin{itemize}
 \item There is no need to lock the target. There is no functionality to set the trigger to unsafe\footnote{In the real Viggen a battery would be activated and the missile would have to be shot within 40 seconds or it would get unusable.}. And there is no specific HUD mode (as in the real Viggen).

 \item Fire with the trigger.
 \item Guide the missile by using one of the following methods:
 \begin{itemize}
  \item Use the keybinding \textbf{\texttt{y}} to toggle the joystick to be a cursor and then use the usual axis to guide the missile. Note: needs to be done after the trigger has been pressed. To be able to use the joystick for the flight controls again, you need to press \textbf{\texttt{y}} again. If the ground collision warning is triggered, then the joystick is automatically set back to control the aircraft again, but you will have lost guidance on the missile.
  \item Bind the following to axis of your joystick or throttle: textbf{\texttt{/controls/displays/cursor-slew-x}} and textbf{\texttt{/controls/displays/cursor-slew-y}}. This makes it possible to have different sinsitivity from the joystick. Note that it is the same binding as e.g. used in the JA 37Di variant --- but the AJS does not have a screen with a cursor (and the JA 37Di does not have the Rb 05A).
 \end{itemize}

\end{itemize}

Tips:
\begin{itemize}
 \item Make sure to fly within the recommended speed of 700 km/h-1150 km/h. You might loose sight of the light at the back of the missile at a distance around 2 km due to some restrictions in FlightGear. And would probably not be able to see the target anyway.
 \item Make sure to fly either streight horisonal or in a light dive. If you have your nose up, you might loose track of the missile and/or target. Note that one of the advantages of this missile is that it can be guided considerable off boresight.
 \item Especially for large objects (factory) make sure to aim in the middle of the object and on gound level. The proximity fuse might otherwise not recognise it.
\end{itemize}

}{}

\chapter{Air to Air}
\ifthenelse{\boolean{AJS}}{
\section{Rb 05A missile}
The Rb 05A could also be used against slow flying air targets like helicpoters or transport planes. See section \ref{Rb05A} on page \pageref{Rb05A} for a description. The difference to ground/naval attack is that the distance had to be measured with the radar.

}{}

\section{Cannon}
\url{https://www.aef.se/Flygvapnet/Tidskrifter/FV_Nytt/Flygvapennytt_1993-2.pdf} has an article and 2 pictures about the automatic aiming mode for the cannon in the JA version (in Swedish).

\appendix
\begin{landscape}
\chapter{Viggen Related Videos}
The following selection of videos has been made based on operational content or visibility of details. There exist many more videos --- some of which with higher quality from recent air shows. If you miss a video below, let us know.

\begin{table}[!th]
\begin{tabular}{|p{5cm}|c|c|c|p{9cm}|}
\hline
Title & Length & Subtitles & Year & Comments \\
\hline

\href{https://www.youtube.com/watch?v=kBq5qA8r4dA}{SF/SH37 \& JA37 VIGGEN at F13 Wing} & 18 min & yes & 2000 & Mostly about the reconnaissance and photo version\\
\href{https://www.youtube.com/watch?v=0sRACNVVmpE}{Saab 37 Viggen 2001 1-2} & 15 min & no & ?? & Historic part plus early versions\\
\href{https://www.youtube.com/watch?v=pAPteuBsRGg}{Saab 37 Viggen 2001 1-2} & 14 min & no & ?? & Mostly about the evolution of the JA. Some footage of HUD and CI/TI screens.\\
\href{https://www.youtube.com/watch?v=fmqXa0oetUA}{VI FLÖG HÅRT OCH STRIDSMÄSSIGT} & 15 min & no & ?? & A fighter pilot tells stories\\
\href{https://www.youtube.com/watch?v=ErK3zaNRccE&}{Jaktviggenpilot på Bråvallajakten} & 14 min & no & 1989 & A day in the live of a fighter pilot\\
\href{https://www.youtube.com/watch?v=qaNt9_sQdGI}{Kurvstrid (Dogfight) SAAB JA37C VIGGEN} & 8 min & yes & ?? & Various dogfights with some HUD visibility\\
\href{https://www.youtube.com/watch?v=oB6-jJEWjWk}{S som i Speed - Göran flyger Viggen} & 6 min & no & 1996 & Some low level flying\\
\href{https://www.youtube.com/watch?v=gwgJNdWZlj0}{SH-37 Viggen Sea Surveillance} & 4 min & n/a & ?? & Some nice sequences of the AJS CI and a few glipse of the HUD\\
\href{https://www.youtube.com/watch?v=mgPS5-SbI0c}{Viggen attackuppdrag} & 2 min & no & ?? & Attack with rockets using pop-up method. Some sequences of the HUD.\\
\href{https://www.youtube.com/watch?v=eQjAp7bTnUg}{Kalla Krigets Fordon avsnitt 6: SAAB 37 Viggen} & 9 min & no & 2016 & A bit of history and relation between the plane and the road base system. During the very last seconds some close-up sequences of the CI.\\
\href{https://www.youtube.com/watch?v=slm9ksxU0HY}{Viggen Road base exercise} & 6 min & n/a & ?? & As the title says.\\
\href{https://www.youtube.com/watch?v=hWrsP3hq5M8}{AJ-37 Viggen Low Level Flying} & 4 min & n/a & ?? & As the title says. Some HUD sequences incl. rocket attack.\\
\href{https://www.youtube.com/watch?v=UOszNlNeRVs}{SAAB 37 Viggen} & 4 min & n/a & ?? & Some nice air ballet plus a few sequences of the CI.\\
\href{https://www.youtube.com/watch?v=IbsYsUvCy7s}{Rörlig klargöring JA37 Arlanda} & 43 min & no & 1996 & Ground operations of JA 37 on an improvised stand at Arlanda airport\\
\href{https://vimeo.com/60091080}{Viggen i Älskar, älskar inte} & 4 min & no & ?? & Excerpts from the feature film, Älskar, älskar inte. Nice dogfighting between JA and AJS plus some CI sequences\\

\hline
\end{tabular}
\caption{Selected videos featuring the Viggen. Unless described otherwise in the comments, the language is Swedish}
\end{table}

% Videos watched but not found interesting enough
% \href{https://www.youtube.com/watch?v=rFZKvO95bQ0}{2018-10-17 Svensk stridspilot under det kalla kriget}
% \href{https://www.youtube.com/watch?v=RHVzkVwI68w}{Vetenskapens Värld - Viggen} Mostly project and politics history until first viggen operational
% \href{https://www.youtube.com/watch?v=M8HF1gH0MPc}{37 Viggen - The Power of Viggen!} Mostly flight display
% \href{https://www.youtube.com/watch?v=Z_EnkvE6LZA}{Fredens Hav}
\end{landscape}

\end{document}
