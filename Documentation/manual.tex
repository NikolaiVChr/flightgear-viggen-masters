\usepackage[english]{babel}
\usepackage[utf8]{inputenc}
\usepackage[T1]{fontenc}

\usepackage{graphicx}
\usepackage{hyperref}
\usepackage{wallpaper}
\usepackage{xcolor}
\usepackage{geometry}

\pagestyle{headings}

\begin{document}
\newgeometry{bottom=2cm}
\ThisCenterWallPaper{1.0}{images/title}
\begin{titlepage}
  \centering
  \sf
  {\Huge Saab \ifAJS{AJS 37}\ifJA{JA 37Di} Viggen}
  \\[1cm]
  {\Huge Flightgear Flight Manual}
  \\[16cm]
  \color{white}
  \emph{model by}\\
  Anders Lejczak, Justin Nicholson, Enrico Castaldi,\\
  Joshua Davidson, Nicola B.\ Bernardelli, Isaac Protiva,\\
  Nikolai V.\ Chr.
\end{titlepage}
\restoregeometry

\currentpdfbookmark{contents}{Contents}
\tableofcontents

\chapter{Introduction}
\currentpdfbookmark{introduction}{Introduction}
\addcontentsline{toc}{chapter}{Introduction}

This guide intends to be a pilot's handbook for the Saab 37 Viggen aircraft modelled for the \href{http://www.flightgear.org}{FlightGear (FG)} flight simulator. The manual assumes acquaintance with FlightGear and does in general not provide any generic help to use the simulator.

The original aircraft's objective was military combat --- and that is also an explicit part of this manual.

The Saab Viggen was originally intended as a multi-role combat aeroplane. However, while largely a sngle type of aircraft/airframe was able to fulfill multiple roles, it was only the Viggen's Swedish successor Saab JAS 39 Gripen that was able to fulfill the requirements for a multi-purpose combat airplane. Consequently, the Viggen type designations reflect their operational roles:
\begin{description}
\item[AJ] Attack, Jakt (surface attack with secondary air defence role)
\item[SK] Skol (conversion training)
\item[SH] Spaning, Havövervakning (reconnaissance, sea surveillance)
\item[SF] Spaning, Foto (reconnaissance, photography)
\item[JA] Jakt, Attack (air defence with secondary attack role)
\end{description}

The variants explicitly modelled and subject to this manual are the \emph{JA 37Di} (featuring some of the glass instrument panels used in the JAS Gripen) and the \emph{AJS 37}, which resulted out of a modification programme providing existing AJ 37, SF 37 and SH 37 variants with a limited multi-role (attack, fighter and reconnaissance) capability.

\section{References}
\subsection{FlightGear Related}
\begin{itemize}
\item The \href{http://wiki.flightgear.org/Saab_37_Viggen}{FG wiki article} contains a feature overview, links to forum articles etc.
\item \href{http://opredflag.com/}{Operation Red Flag (OPRF)} is a FlightGear military simulation community discussing, developing and using many of the combat features in the Viggen.
\item On Discord there is a dedicated \href{https://discord.gg/jc5pSM5}{Viggen server} and a \href{https://discord.gg/SmGFnJN}{OPRF server}.
\item The \href{https://www.youtube.com/playlist?list=PLogi97V-ki0GfCLqimTtIq9RIVcm-GRFE}{Flightgear Saab 37 Viggen YouTube Playlist} includes amongst others a set of tutorial videos featuring the JA and the AJS variant.
\end{itemize}

\subsection{Other Combat Flight Simulators}
Both \href{https://www.digitalcombatsimulator.com/en/index.php}{DCS} and \href{https://www.benchmarksims.org/}{BMS Falcon} have flyable Viggens. Especially for DCS there is an abundance of resources available on the internet (e.g. YouTube). Please note that while there can be valuable additional content, the capabilities modelled will differ.

\subsection{Swedish Documentation Sites}
\begin{itemize}
\item Arboga Elektronikhistoriska Förening: \url{https://www.aef.se}, e.g. 
\begin{itemize}
\item \href{https://www.aef.se/Avionik/Notiser/PS-37/PS-37A.htm}{Spanings- och siktesradar PS-37/A (for the AJS)}
\item \href{https://www.aef.se/Avionik/Notiser/Siktesradar_PS-46_1.htm}{Flygradarinstallation JA37 ---Siktesradar PS46} (includes links to 2 articles from Ericsson)
\item \href{https://www.aef.se/Flygvapnet/Tidskrifter/FV_Nytt/FVN_oversikt.htm}{FlygvapenNytt 1960 – 2003}: \url{https://www.aef.se/Flygvapnet/Tidskrifter/FV_Nytt/Flygvapennytt_1991-2.pdf} announced the AJS Viggen and \url{https://www.aef.se/Flygvapnet/Tidskrifter/FV_Nytt/Flygvapennytt_1994-4.pdf} had follow-up articles.
\end{itemize}
\item Digitalt Museum: \url{https://digitaltmuseum.se/}:
\begin{itemize}
\item Aeroseum Göteborg: \url{https://digitaltmuseum.se/owners/S-AER}
\item Flygvapenmuseum: \url{https://digitaltmuseum.se/owners/S-FV}
\item Teknikland: \url{https://digitaltmuseum.se/owners/S-TL}
\item Sveriges militärhistoriska arv: \url{https://digitaltmuseum.se/owners/S-MHA}
\end{itemize}
\end{itemize} 

\chapter{FlightGear Download and Installation}
The JA-37Di and the AJS-37 variants of the Viggen are downloaded and installed as one set. You can choose between two options to download and install:
\begin{itemize}
\item The \href{https://github.com/NikolaiVChr/flightgear-saab-ja-37-viggen}{Git repository} has the latest version of the Viggen and is the preferred way to keep up with the constant evolution of the FlightGear model. If you are using the Git repository, make sure to add the correct directory path\footnote{E.g. /home/pingu/fg\_aircraft/flightgear-saab-ja-37-viggen/Aircraft} to the \emph{Additional aircraft folders} section in tab \emph{Add-ons} in the FlightGear launcher.
\item FlightGear's official hangar \href{http://wiki.flightgear.org/FGAddon}{FGAddon} is peridically updated with stable versions from the Git version. 
\end{itemize}

The guide you are reading is keeping up with the latest Git version and therefore the features available in the FGAddon version might be different. Unless you have a good reason not to do so, please consider using the Git version and keep it updated.


In the \emph{Aircraft} tab in the Launcher make sure to choose the correct variant for your next flight.

\chapter{FlightGear Menu Entries}
\section{Help}
The Help menu contains 2 menu items, which are Viggen specific:
\begin{itemize}
\item Aircraft Help: the top section contains the most important key bindings for the Viggen\footnote{The displayed list is only a selection. For the total list you need to delve into the installation directory and look into flightgear-saab-ja-37-viggen/Aircraft/JA37/Systems/ja37-input.xml.}. The lower section contains a set of tipps to get your started with flying.
\item Aircraft Checklists: A set of interactive checklists especially for start-up, taxi, take-off and landing.
\end{itemize}

\section{AJS-37}
This menu contains a set of menu items, which ease some configurations, which are not available otherwise or would be tedious to do.
\begin{itemize}
\item Select Livery
\item Auto start/stop: Lets you start and stop the plane. The progress is shown in the top centre of the screen in blue text. \glqq Engine ready\grqq is the final notification of the start-up sequence. The shut-down sequence is done, when the aircraft is dark.
\item Repair
\item Loadout: Opens a custom dialog to allow fast selection of a suitable loadout. This is an alternative to menu item Fuel and Payload in menu Equipment. It will make sure that you only select weapon combinations, which were valid in the real Viggen.
\item Refule: Standard: all tanks full, but no external centre tank
\item Refule: Long distance: adds fuel to the drop tank on the centre fuselage pylon.
\item Refule: Test flight: reduced fuel amount to ca. 80\% of the standard
\item Performance Monitor
\item Systems Monitor
\item Toggle external power: note that this is one of the things added and removed automatically when using the auto start/stop function.
\item Options: various configuration options
\item MP Options: use this dialog to enable/disable weapons messages over MP.
\end{itemize}

\section{View}
In menu item \emph{Rendering Options} you should preferably enable \emph{Light Scattering (ALS)} on the right side in \emph{Shader Effects}. Select \emph{Custom settings (fine tuning)} and in \emph{Shader Options} set the slider related to \emph{Model} to the right side (highest setting). This is amonst otheres needed to get nice flood lights in the cockpit at night.

The \emph{View options} menu item allows to enable tooltips on mouse over / mouse press for gauges and knobs.

\part{Aircraft Presentation}
This manual does not contain a generic presentation of the real Viggen. There are many books and internet resources, which do a good job for this. A comprehensive book in English covering all variants is \glqq Saab 37 Viggen --- The ultimate portfolio\grqq by Jan Jørgensen (published in 2014 by \href{http://www.nordicairpower.com/}{Nordic Airpower}).

What is presented her is how the aircraft is modelled in FlightGear.

\chapter{Aircraft Overview}
\chapter{Controls and Instrumentation}
\chapter{Systems}

\part{Flight Operation}
\chapter{Standard Procedures}
\chapter{Emergency Procedures}

\part{Weapons Operation}
\chapter{Ground and Sea Attack}
\chapter{Air to Air}
\section{Cannon}
\url{https://www.aef.se/Flygvapnet/Tidskrifter/FV_Nytt/Flygvapennytt_1993-2.pdf} has an article and 2 pictures about the automatic aiming mode for the cannon in the JA version (in Swedish).

\end{document}
